
\documentclass[../report]{subfiles}
\begin{document}
\part{Implémentation}



\chapter{Lecture d'article}
    % sec difficulté interprétation des sources
\chapter{Clarté de la documentation des modules rust}

\chapter{Le protocole}

\section{Protocole à 2 tours}

Pour permettre la confidentialité du vote y compris envers le serveur central receptionnant
le bultin de vote, le protocole de vote fonctionne en 2 tours.
Lors du premier tours chaque participant publie une clé publique par candidat/option et une 
preuve NIZK\footnote{Non-interactive zero-knoledge} que ces clés sont bien formée.
Au début du second tours, chaque participant vérifie toutes les preuves que les clés publiques
sont bien formée.
Le bulletin de vote est ensuite formé à partir de l'ensemble des clés publiques des autres 
votants et du choix de vote effectué.
Avec le bulletin de vote, chaque participant publie une preuve NIZK que son vote est bien 
une permutation des choix possible.

\section{Calcul du résultat}\label{sec:res:proto:resultat}

\todo{Est-ce log discret sur une courbe éliptique? est-ce que je dis la somme ou le produit? }
Le calcul du résultat final est le calcul du logarithme discret sur le produit de tout les 
bulletins de vote.
Ce calcul est un problème difficile sur de grand nombre, cepandant pour des nombres 
relativements petits, une recherche par brute force est possible.
Le temps de calcul est parfaitement raisonable et fonctionnel, même pour de très grand groupe, 
une analyse détaillée est fait dans la \aaref{sec:res:perf:resultat}.


\end{document}
