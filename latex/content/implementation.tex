
\documentclass[../report]{subfiles}
\begin{document}
\part{Implémentation}

\chapter{Cas d'utilisations}

De manière générale l'utilisation du vote électronique par rapport à un vote traditionel ouvre la voie à 
des attaques qui était jusqu'à présent impossible. 

Lorsque l'action de voter ne se fait plus dans un isoloire dans un bureau de vote (ce qui est déjà le cas
pour le vote par correspondance suisse) il y a une plus grande possibilité de coércition des votants.
On ne pourra jamais exclure que des votants aient été contraint dans leur choix. 
Le vote électronique ne résouds pas ce problème.

\section{Cas d'utilisations}
\subsection{Vote lors d'une assemblée}

Lors d'un assemblée d'une association ou d'une entreprise/coopérative, les décisions sont souvant prise avec
des votations sur des propositions.
C'est votations peuvent être faites par e-voting dans la majorité des cas, cela permet de limité les erreurs 
de comptage et d'avoir un résultat plus rapide s'il y a beaucoup de monde.
Le vote électronique permet également d'avoir une plus grande confiance dans les résulats, car on ne fait pas 
confience à un petit groupe chargé du décompte des voix (groupe qui ne contient pas toujours de personne de 
chaque position possible).

Le corruption des personnes chargées du décompte peux être facile, alors que si l'application utilisé pour
l'e-voting est externe au groupe, il peut être beaucoup plus difficile de les corrompre.

\begin{important}{Vote d'importance}
  Même pour de petite structure, il peut avoir des risques importants à utilisé un système de vote électronique, 
  notament s'il y a un ou des oppansants puissants (économiquement, politiquement ou informatiquement), et ce que
  se soit par rapport à un vote spécifique ou de manière globale à la structure.
\end{important}

\todo{Autre cas d'utilisation?}
%\section{Sondage rapide et anonyme au sein d'un groupe}
%
%Lorsqu'on fait des sondages, il peut être difficile de garantir l'anonymat des réponses. 
%S'il y a besoin d'anonymat et de rapidité, l'utilisation d'une solution de e-voting peut permettre d'organiser 
%de tel sondage. 

.

\section{Cas de NON-utilisation}

Les cas suivants ne devrait pas utilisé de vote électronique pour leurs votations. 
Dans le cas où il l'utiliserait quand même, une analyse détaillée des risques doit être fait au préalable.

\subsection{Élection à l'échel d'un pays}\label{sec:impl:non-use:pays}
Utiliser le vote électronique pour une élection ou une votation d'un pays n'est pas une bonne idée.
Il y a toujours des risques dû à l'utilisation de logiciel informatique, et quand bien même le logiciel de 
votation serais exempt de vulnérabilité il y aurait toujours des menaces grâve qui peserais sur la votation.
\begin{itemize}
  \item des attaques visants les ordinateurs des votants
  \item la circulation de fausse application de vote
  \item des attaques visants Internet ou le réseau utiliser pour voter
\end{itemize}

La vie politique d'un pays ne doit pas pouvoir être contrôler par un petit groupe de personne, sans accord 
explicite de toute la population.

\subsection{Votation d'importance stratégique}
Lorsqu'une votation a une certaine importance stratégique (pour l'entreprise, la collectivité ou l'organisation qui l'organise)
et que l'on peux suspecter qu'il y a des personnes externe qui aient un intérêt pour une options ou une 
autre,\footnote{Que cette personne existe ou non, le simple fait que l'on ne soit pas sûre est suffisant}
cette votation ne devrait donc pas être faite via de l'e-voting. 
Les vulnérabilités non-mitigeable sont les même qu'à la \aaref{sec:impl:non-use:pays}


\chapter{Lecture d'article}
    % sec difficulté interprétation des sources
\chapter{Clarté de la documentation des modules rust}

\end{document}
