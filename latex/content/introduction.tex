\documentclass[../report]{subfiles}
\begin{document}
  \chapter{Introduction}

  \section{Définition}

  \paragraph{Vote électronique ou e-voting}
  Vote effectuer grâce à des outils informatique.
  Ce terme peut à la fois désigner un vote effectuer à distance via Internet ou un vote effectuer
  dans un bureau de vote à l'aide d'une machine à voter.
  Dans ce document nous nous focalisons sur le vote électronique à distance.
  Lorsque ce n'est pas préciser, nous parlons du vote électronique à distance et non pas dans un bureau de vote.

  \paragraph{Comptage électronique}
  \todo{}
  
  \paragraph{NIZK proof of knoledge}
  \todo{}

  \paragraph{Élection ou votation}
  Une élection est le fait qu'un groupe de personne (les électeurices) choississent une (ou plusieurs) personne
  pour les représenter ou pour toutes autres tâche.
  Une votation est le fait qu'un groupe de personne (les votant·es) choissient une (ou des) options.
  Une élection est donc simplement une votations ou les options possibles sont des personnes.
  Ce document ne précise pas à chaque fois que ce qui est dit pour l'un est également valable pour l'autre de 
  manière générale.



  \section{Type de méthode de scrutin analyser}
  \section{candidat vs options}
  \section{Vote blanc/nuls/abstentions dans les simulations}
  \section{Exemple fictif}
  \section{referendhum cas particulier d'une élection}

  \section{Définition et abréviation}
  
  \begin{description}
    \item[vote utile] \aref{diff:comp:util:util}
    \item[quorum]
  \end{description}

  \todo{Alice est la votante dans les exemples, présenté un exemple de structure utilisé dans 
  le document}

\end{document}

