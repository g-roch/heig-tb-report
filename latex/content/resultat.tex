
\documentclass[../report]{subfiles}
\begin{document}
\part{Résultat}

\chapter{Test des perf}


\begin{nota}{Estimation}
	Les valeurs estimées sont préfixée par \textasciitilde{}, plus d'information à la \aaref{sec:intro:notation:estimation}
\end{nota}

Les analyses ci-après on été effectué sur un ordinateur de bureau (i7-6700K, 8~cœurs, 8~Gio de RAM).
Les programmes de test on été écrit et exécuté sur un seul thread, donc seul environ $\frac{1}{8}$ de 
la capacité du processeur a été utilisé.
Durant les tests la fréquence effective du processeur s'est stabilité à 4 GHz.

Dans le tableau suivant, voici le nombre de votant pris en compte pour les simulations de ce chapitre.

\begin{table}[H]
  \begin{center}
    \begin{tabular}{|c|l|r|l|}
      \hline
       & Groupe de personne                          & Nb de votant·es & Source \\
      \hline
      \hline
      $ASSO$ & Petit association                  & 300 & \\
      $LS_p$ & Population lausanoise [2020]       &         140~202 & \href{https://fr.wikipedia.org/wiki/Lausanne}{Wikipédia français: Lausanne}\\
      $MIG_v$ & Migros: ayant voté [2022]         &         630~000 & \href{https://www.rts.ch/info/suisse/13176954-la-vente-dalcool-reste-interdite-a-la-migros-apres-un-non-massif-en-votation.html}{RTS Info, 16 juin 2022}\\
      $MIG_i$ & Migros: coopérateurs [2021]       &       2~280~000 & \href{https://corporate.migros.ch/dam/jcr:d5b1ae79-a8e8-4b98-9f2d-4b4974d39486/faits-et-chiffres-2021.pdf}{Migros Faits et chiffres 2021}\\
      $FR_v$ & Français ayant voté [2022]         &      35~923~707 & \href{https://fr.wikipedia.org/wiki/Élection_présidentielle_française_de_2022}{Wikipédia: élection présidentielle fr 2022}\\
      $FR_i$ & Électeurs français inscrits [2022] &      48~747~876 & \href{https://fr.wikipedia.org/wiki/Élection_présidentielle_française_de_2022}{Wikipédia: élection présidentielle fr 2022}\\
      $FR_p$ & Population française [2020]        &      68~014~000 & \href{https://fr.wikipedia.org/wiki/France}{Wikipédia français: France}\\       
      $EU_p$ & Population continent européen      &     743~000~000 & \href{https://fr.wikipedia.org/wiki/Europe}{Wikipédia français: Europe}\\       
      $TER_p$ & Population mondiale [2021]        &   7~874~966~000 & \href{https://fr.wikipedia.org/wiki/Population_mondiale}{Wikipédia français: Population mondiale}\\       
      \hline
    \end{tabular}
  \end{center}
  \caption{Nombre de votant·es possible par groupe de population}\label{tab:}
\end{table}

\section{Taille de données}

Les données publiques générées lors du premier tours corresponds à 88 B par électeur et par option.
Lors du second tours 104 B suplémentaire sont publié par électeur et par option.
Les données totals sont donc de 192 B par électeur et par option.

Nous pouvons constater dans le \aaref{tab:res:perf:size} que la quantité de données générée lors du premier tour et
que chaque votant a besoin de télécharger pour le second tour est resonable pour de groupe de quelques centaines de personnes.
Dès que la taille atteint une taille proche d'une ville (Lausanne dans le tableau) la taille deviens importante et peut poser
problème pour certains votants s'ils n'ont pas une bonne connexion Internet.

\todo{update}

\begin{table}[H]
  \begin{center}
    \begin{tabular}{|c|r|r|r|r|}
      \hline
       & Nb de votant·es      & 1\up{er} tour                & 2\up{nd} tour & total\\
       & $n$                  & $n \cdot o \cdot 88\text{B}$ & $n \cdot o \cdot 104\text{B}$  &  \\
      \hline
       &                      & $o = 30$ & $o = 30$ &                    \\
      \hline
      \hline
      $ASSO$  &           300 & 257,8 kiB & 307,7 kiB & 562,5 kiB \\
      $LS_p$  &       140~202 & 117,7 MiB & 139,1 MiB & 256,7 MiB \\
      $MIG_v$ &       630~000 & 528,7 MiB & 624,8 MiB &   1,1 GiB \\
      $MIG_i$ &     2~280~000 &   1,9 GiB &   2,2 GiB &   4,1 GiB \\
      $FR_v$  &    35~923~707 &  29,4 GiB &  34,8 GiB &  64,2 GiB \\
      $FR_i$  &    48~747~876 &  40,0 GiB &  47,2 GiB &  87,2 GiB \\
      $FR_p$  &    68~014~000 &  55,7 GiB &  65,9 GiB & 121,6 GiB \\
      $EU_p$  &   743~000~000 & 608,9 GiB & 719,7 GiB &   1,3 TiB \\
      $TER_p$ & 7~874~966~000 &   6,3 TiB &   7,4 TiB &  13,8 TiB \\
      \hline
    \end{tabular}
  \end{center}
  \caption{Temps de calcul des résultat sans optimisation}\label{tab:res:perf:size}
\end{table}


\section{Vérification des NIZK du premier tour}
\section{Vérification des NIZK du second tour}

\section{Calcul du résulat final}\label{sec:res:perf:resultat}

Comme décrit dans la \aaref{sec:res:proto:resultat}, le calcul du résultat final est la résolution d'un logarithme discret (sur une courbe éliptique).
Ce problème étant connu pour être difficile, voici quelques analyse de brute force pour utilisant la bibliothèque utilisé
pour l'implémentation de ce projet.
La valeur recherchée dépands de 3 paramètres: le nombre de votant·e, le nombre d'option disponible et la répartition des votes.
Les deux premiers paramètres sont facile à estimé ou à défaut d'avoir une borne supérieur.

Le valeur recherchée pour chaque candidat est la somme des votants pondérer par le classement de ce dit-candidat par le votant.
Par exemple pour 3 options :

\[
  res = 1 \cdot \sum votants_{\text{pos = 1}}
      + 2 \cdot \sum votants_{\text{pos = 2}}
      + 3 \cdot \sum votants_{\text{pos = 3}}
\]

Le score maximume d'un candidat est donc (si l'ensemble de la population le classe en premier)

\[
  score = votants * options 
\]

Au dela de 30 options, le classement de toutes les options par les votants devient très compliqué.
Nous estimons donc que 30 options est une borne supérieur pour nos calcul.

Deux algorithmes ont été écrit pour tester la performence.
Le premier (\aaref{lst:perf:resultat-1}) calcul à chaque itération la multiplication du générateur 
par $i$, on obtient ainsi un calcul de 20~000 essaies par seconde.
Un second algorithme (\aaref{lst:perf:resultat-2}) a été écrit pour essaier d'accélerer le 
traitement, celui-ci garde en mémoire le résultat de l'itération précédente pour lui ajouter le générateur.
Il obtient un calcul de 2~400~000 essaies par seconde.


\begin{table}[H]
  \begin{center}
    \begin{tabular}{|c|r|r|r|r|}
      \hline
       & Nb de votant·es      & points max  & durée algo~\ref{lst:perf:resultat-1}           & durée algo~\ref{lst:perf:resultat-2}           \\
       & $n$                  & $n \cdot o$ & $\frac{n \cdot o}{v_1}$ & $\frac{n \cdot o}{v_2}$ \\
      \hline
       &                      & $o = 30$    & $v_1 = 20~000$         & $v_2 = 2~400~000$         \\
      \hline
      \hline
      $ASSO$  &           300 &           9~000 &                                0''           &                            0''           \\
      $LS_p$  &       140~202 &       4~206~060 &                             3'30''           &                            2''           \\
      $MIG_v$ &       630~000 &      18~900~000 & \textasciitilde{}          15'45''           &                            8''           \\
      $MIG_i$ &     2~280~000 &      68~400~000 & \textasciitilde{}          57'00''           &                           29''           \\
      $FR_v$  &    35~923~707 &   1~077~711~210 & \textasciitilde{}      14h 58'\phantom{06''} & \textasciitilde{}       7'29''           \\
      $FR_i$  &    48~747~876 &   1~462~436~280 & \textasciitilde{}      20h 18'\phantom{42''} & \textasciitilde{}      10'09''           \\
      $FR_p$  &    68~014~000 &   2~040~420~000 & \textasciitilde{}   1j  4h \phantom{20'21''} & \textasciitilde{}      14'10''           \\
      $EU_p$  &   743~000~000 &  22~290~000~000 & \textasciitilde{}  12j 21h \phantom{35'00''} & \textasciitilde{}   2h 34'\phantom{48''} \\
      $TER_p$ & 7~874~966~000 & 236~248~980~000 & \textasciitilde{} 136j 17h \phantom{14'09''} & \textasciitilde{} 1j 3h \phantom{20'37''} \\
      \hline
    \end{tabular}
  \end{center}
  \caption{Temps de calcul des résultat sans optimisation}\label{tab:}
\end{table}

Ces deux algorithmes sont facilement parralélisable. 
On peut partir du principe que la majorité de la population a à disposition un processeur à 4 cœurs

\begin{table}[H]
  \begin{center}
    \begin{tabular}{|c|r|r|r|r|}
      \hline
       & Nb de votant·es      & points max  & durée algo~\ref{lst:perf:resultat-1}           & durée algo~\ref{lst:perf:resultat-2}           \\
       & $n$                  & $n \cdot o$ & $\frac{n \cdot o}{\heartsuit \cdot v_1}$ & $\frac{n \cdot o}{\heartsuit \cdot v_2}$ \\
      \hline
       &                      & $o = 30$    & $v_1 = 20~000$         & $v_2 = 2~400~000$         \\
       &                      &             & $\heartsuit = 4$       & $\heartsuit = 4$          \\
      \hline
      \hline
      $ASSO$  &           300 &           9~000 & \textasciitilde{}              0''           & \textasciitilde{}           0''           \\
      $LS_p$  &       140~202 &       4~206~060 & \textasciitilde{}             53''           & \textasciitilde{}           0'' \\
      $MIG_v$ &       630~000 &      18~900~000 & \textasciitilde{}           3'56''           & \textasciitilde{}           2'' \\
      $MIG_i$ &     2~280~000 &      68~400~000 & \textasciitilde{}          14'15''           & \textasciitilde{}           7'' \\
      $FR_v$  &    35~923~707 &   1~077~711~210 & \textasciitilde{}       3h 44'\phantom{31''} & \textasciitilde{}        1'52'' \\
      $FR_i$  &    48~747~876 &   1~462~436~280 & \textasciitilde{}       5h 04'\phantom{40''} & \textasciitilde{}        2'32'' \\
      $FR_p$  &    68~014~000 &   2~040~420~000 & \textasciitilde{}       7h 05'\phantom{05''} & \textasciitilde{}        3'33'' \\
      $EU_p$  &   743~000~000 &  22~290~000~000 & \textasciitilde{}   3j 05h \phantom{23'45''} & \textasciitilde{}       38'42'' \\
      $TER_p$ & 7~874~966~000 & 236~248~980~000 & \textasciitilde{}  34j 04h \phantom{18'32''} & \textasciitilde{}    6h 50'09'' \\
      \hline
    \end{tabular}
  \end{center}
  \caption{Temps de calcul des résultat sans optimisation}\label{tab:}
\end{table}

\todo{Paragraphe sur la comparaison du temps de calcul et de verification (NIZK + ici) du résulat final}
\todo{Paragraphe sur le calcul du résultat pour tout les candidats}

\begin{lstlisting}[language=Rust,caption={Vitesse de calcul résultat final (très naif)},style=numbers,label={lst:perf:resultat-1}]
fn main() {
    use curve25519_dalek::constants as dalek_constants;
    use curve25519_dalek::scalar::Scalar;
        
    let g = &dalek_constants::RISTRETTO_BASEPOINT_POINT;
    let v = dalek_constants::RISTRETTO_BASEPOINT_POINT;
    for i in 2u64.. {
        if Scalar::from(i) * g == v {
            eprintln!("Result is {i}");
            break;
        }

        // hidden: affichage du temps de calcul à interval régulier
    }
}
\end{lstlisting}%
\begin{lstlisting}[language=Rust,caption={Vitesse de calcul résultat final (un peu moins naif)},style=numbers,label={lst:perf:resultat-2}]
fn main() {
    use curve25519_dalek::constants as dalek_constants;
    use curve25519_dalek::scalar::Scalar;

    let g = &dalek_constants::RISTRETTO_BASEPOINT_POINT;
    let v = dalek_constants::RISTRETTO_BASEPOINT_POINT;
    let mut i = 2u64;
    let mut a = Scalar::from(i) * g;
    loop {
        a += g;
        i += 1;
        if a == v {
            eprintln!("Result is {i}");
            break;
        }

        // hidden: affichage du temps de calcul à interval régulier
    }
}
\end{lstlisting}%

\chapter{Vulnérabilité}

\section{BB compromis}

\subsection{le votant ne sais pas si sont bulletin est bien pris en compte}
\subsection{Verificateur ne sais pas si l'ensemble as été remplacer}

\section{Ordinateur du votant compromi}

\section{Authentification du votant}

\chapter{Retour sur les objectifs}

\section{Seules les personnes autorisées peuvent participer (et qu'une seule fois)}

Pout cette objectif (objectif 2 de la section~\aref{ssec:cdc:cdc:objprinc}),
il faut identifier le votant, ce que mon implémentation ne fait pas encore, il se base 
uniquement sur la bonne fois de chaqu'un lorsqu'ils annoncent leur numéro de votant.
Cependant l'authentification du votant doit ce faire avec une méthode qui permet de connaître
le droit de vote de la personne, de multiple technique existent pour cela.

Ce qui est faisable à moindre coût serais d'utiliser un service OAuth et d'utiliser l'identifiant
d'utilisateur retourné par OAuth comme identifiant de votant
Cette méthode permet (du moment que le serveur n'est pas compromis, lors de la phase 1) de permettre
à tout le monde de vérifier qui a voté.

Chaque bulletin de vote est lié au numéro de votant. 
Cette information étant public, il est trivial de vérifier que chaque identifiant n'est présent
qu'une seul fois dans les bulletin de vote final, pour s'assurer que personne n'a voter de 
multiple fois.
Cependant si le serveur est compromis lors du premier tour de vote, un votant pourrait très bien
générer plusieurs clé de vote associer à des numéro de votant fantaisiste et donc voter 
plusieurs fois.

\section{Seul le votant connait son vote}

Effectivement dans le programme implémenter seul le votant connait sont vote, ni le serveur, 
ni les entités devant calculés le résultat de la votation ne le connait.
La seul manière de connaître le vote d'une personne et de connaître le vote de l'ensemble 
des autres votants.\footnote{Il suffit de soustraire l'ensemble des autres vote au résultat
pour obtenir cette information, mais tout système de vote où les résultats sont publique est
sujet à ce problème}

\section{Le votant n'a pas moyen de prouver ce qu'il a voté}
\todo{}
\section{Chacun peut vérifier que son vote a bien été pris en compte}
\todo{}

\chapter{Conclusion}
\todo{}

\section{résumé++}
\section{dificulté}
%\chapter{Lecture d'article}
% sec difficulté interprétation des sources
%\chapter{Clarté de la documentation des modules rust}
\section{la suite}



\end{document}

