\documentclass[../report]{subfiles}
\begin{document}


  \part{Analyse des différents type de scrutins}
  
  \chapter{Les scrutins}

  \section{Scrutin proportionnel plurinominal}
  
  \subsection{Fonctionnement}
  
  Les citoyens votent pour une liste (de candidats) de leur choix.
  Chaque liste remporte un nombre de siège proportionnel au nombre de voix reçue.
  \cite{noauthor_scrutin_2021}
  
  \subsection{Intégrale ou avec seuil}
  \todo{définir: quotient électoral}
  En proportionnelle intégrale, chaque liste atteignant le quotient électoral, obtient
  le droit à un/des sièges.
  Un seuil de nombre de voix peux être défini pour qu'une liste ait droit à un siège.
  
  \subsection{Préférentiel ou listes bloquées}
  En listes bloquées, les sièges sont répartis au sein des listes par ordre d'apparision 
  des candidats.
  En préférentiel, les sièges sont répartis en fonction des préférences des électeurs.
  En changeant l'ordre des candidats ou en biffant/rajoutant certain ou en cumulant certain 
  candidat de multiple foix.
  
  \subsection{Répartition des sièges}
  Le siège ne pouvant pas être divisé, une méthode doit être définit pour répartir 
  correctement ces fragments de siège.
  
  \section{Scrutin uninominal majoritaire}
  
  Chaque votant vote pour son candidat/option préférée, et l'option/candidat avec le plus de
  voix l'emporte.
  
  \subsection{Majorité relative ou absolue}
  \subsubsection{Majorité absolue}
  Dans le cas d'un scrutin à majorité absolue, l'option avec plus de 50\% de vote est choisie.
  Si on exclues le cas d'un ex-equo et que l'on ne prends pas en compte les vote 
  blancs/nuls/abstansions, pour 2 options possibles, il y a forcement un gagnant.
  
  Dans le cas où aucune options n'atteint les 50\%, une tour supplémentaire est généralement
  organiser.
  Ce tour supplémentaire peux être à la majorité relative dans une volonté de limité le nombre
  de tour.
  
  \begin{nota}{Pourquoi 50\%~?}
    Pour certain scrutin plus important, le seuils est plus élevé que 50\%, example 60\% ou 70\%. 
    En quoi est-ce que 50\% est un bon choix, est-ce que ça veux dire que le choix de la moitié
    des votants n'as pas importance~? 
  \end{nota}
  
  \subsubsection{Majorité relative}
  Dans le cas d'un scrution à majorité relative, l'option avec le plus de voix est choisie, 
  quelque soit son pourcentage de voix.
  
  \subsection{Nombre de tour}
  
  Dans la pluspart des cas, il y a plusieurs tours d'organiser, soit jusqu'à ce qu'il y a 
  un candidat qui remporte une majorité absolue, soit un nombre prédéfini de tour (souvant 2).
  Habituellement, entre chaque tour 1 ou plusieurs candidats sont éliminés.
  
  \begin{nota}{Qui éliminer ?}
    Habituellement, les candidats/options avec le moins de voix sont éliminé, mais
    est-ce vraiment leurs électeurs qui sont le plus suceptible de changer d'avis ?
    Est-ce qu'un électeur d'un des 2 candidat proche mutuellement n'aurait pas plus
    de facilité à changer son vote ?
  \end{nota}
  
  \section{Méthode Borda}

  \subsection{Justification choix possible}

  Dans le cas où un vote partielle est autorisé/possible (c.-à-d. certaines options sont 
  non-classées).
  Cela pourrait être bien de ne pas scoré les options (de la meilleures au pire)
  avec les valeur ($n$, $n-1$, ..., $1$) et les options non-classée à $0$.
  C'est mieux de les scorés du pire au meilleure ($1$, $2$, ..., $j$) avec $j$ le nombre
  d'option classée et laisser à $0$ les options non-classée.
  Ce choix évite de perdre les avantages de la méthode Borda si un grande partie des votants
  ne classe que 1 ou 2 candidats.
  \cite{emerson_original_2013}
  
  \section{Méthode de Condocet}
  \todo{}
  \section{Jugement majoritaire}
  \todo{}
  \section{Méthode de Coombs}
  \todo{}
  \section{Vote alternatif}
  Aussi appelé vote préférencielle transférable ou Méthode de R
  \todo{}
  \section{Vote cumulatif}
  \todo{}
  \section{Vote par abbrobation}
  \todo{}


  \chapter{Problème et avantage}
  
  \todo{Prendre en compte ou non les votes blancs/nuls et les abstentions}
  \todo{$50 \% + 1$ versus $ > 50\%$}
  \todo{Avantage: il y a de la réflexion entre les tours lors de scrutin 
  multi-tours. Tout le monde ne reste pas forcement sur ses positions}

  
  %\begin{nota}[by me]{Ma note}\lipsum[1][1-5]\end{nota}
  %\begin{question}[by me]{Ma question}\lipsum[2][1-5]\end{question}
  %\begin{important}[by me]{Ma not imp!}\lipsum[3][1-5]\end{important}
  %\begin{warning}[by me]{My warn}\lipsum[4][1-5]\end{warning}

\end{document}
